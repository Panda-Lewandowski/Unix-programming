% !TEX root = main.tex
\section{testCIO.c}

\paragraph{Текст программы}\hfill\\

\biglisting{../../testCIO.c}


\paragraph{Вывод}\hfill\\

\biglisting{../../testCIO_output.txt}

\paragraph{Анализ}\hfill\\
 С помощью системного вызова $open()$ создается дискриптор файла, файл открывается только на чтение, указатель устанавливается на начало файла. Если системный вызов завершается успешно, возвращенный файловый дескриптор является наименьшим, который еще не открыт процессом. В результате этого вызова появляется новый открытый файл, не разделяемый никакими процессами, и запись в системной таблице открытых файлов. 
 
 Далее функция $fdopen()$ связывает два потока с существующим дискриптором файла. Функция $setvbuf()$ изменяет тип буферизации на блочную (полную) размером в 20 байт. 
 
 В цикле осуществеляется чтение из потоков и вывод в $stdout$с помощью системных функций $fscanf, fprintf$. Флаги $flag1, flag2$ изменят свое значение с 1 на -1 тогда, когда число прочитанных символов станет равно нулю. Стоит помнить о том, что открытые файлы, для которых используется ввод/вывод потоков, буферизуются. Т.к. размер буфера установлен в 20 байт, по факту в $buff1$ помещается строка $Abcdefghijklmnopqrst$, а в $buff2$ - $uvwxyz$. В результате поочередного вывода из каждого буфера потока получим строку выше.

